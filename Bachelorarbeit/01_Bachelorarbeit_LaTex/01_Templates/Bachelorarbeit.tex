\documentclass[12pt]{report}
\usepackage[utf8]{inputenc}
\usepackage{a4}
\usepackage[none]{hyphenat} %hyphenation
\sloppy
\usepackage{parskip} %no indentation after paragraphs
\usepackage{umlaute}
\usepackage{afterpage} %for using \afterpage{\clearpage} (don't push images to the end of a chapter)
\usepackage{makeidx}
\usepackage[numbers]{natbib}
\usepackage{graphicx}
\usepackage{picins} %provides precise control over the placement of inline graphics
\usepackage{setspace}
\usepackage{titlesec}
\usepackage{dsfont} %math symbols
\usepackage{tabularx}
\usepackage{floatflt} %float text around figures and tables
% Florian Schulze, 06.06.2012
% v1.0, latest edit: 06.06.2012

\usepackage{enumitem} %resume counting from previous enumerate block
\usepackage{amsmath,amssymb}
\usepackage[format=default,font=footnotesize,labelfont=bf]{caption}
\usepackage{listings} %for listing source code
\usepackage{color}
\usepackage{algpseudocode} %for listing pseudocode
\usepackage{algorithm} %wrap algpseudocode and enrich with label etc.
\usepackage{float} % for [H] after floats


\graphicspath{{C:/Users/Kevin/Bachelarbeit/Bachelorarbeit/01_Bachelorarbeit_LaTex/02_Figures/}}
\titleformat{\paragraph}[hang]{\normalfont\bfseries}{\theparagraph}{.5em}{}


\makeindex
\frenchspacing
\sloppy

\pagestyle{headings}

\textwidth16cm
\textheight22cm

\topmargin0cm
\oddsidemargin0cm
\evensidemargin0cm


\newcommand{\bildklein}[3]{  
	\begin{figure}[hp]
	\begin{center}
	\includegraphics[width=0.5\textwidth]{#1}
	\end{center}
	\caption[#2]{#3}
	\end{figure}
}
  	
\newcommand{\bildgross}[3]{  
	\begin{figure}[hp]
	\begin{center}
	\includegraphics[width=0.95\textwidth]{#1}
	\end{center}
	\caption[#2]{#3}
	\end{figure}
}
  

\newcommand{\eqn}[3]{
	\begin{figure}[hp]
	\begin{equation}#1\end{equation}
	\caption[#2]{#3}
	\end{figure}
}

\input{figures/tumlogo}

\begin{document}

\nocite{*} %include uncited references in bibliography
\hoffset=5mm
\thispagestyle{empty}

\begin{center}
	\bigskip \bigskip \bigskip 
	\oTUM{6.0cm} \\
	\vspace*{0.8cm}
	{\huge \bf Technische Universität} \\
	\bigskip
	{\huge \bf München} \\
	\bigskip \bigskip \bigskip
	{\huge \bf Fakultät für Informatik} \\
	\bigskip \bigskip \bigskip
	{\Large \bf Master's Thesis in Informatik} \\
	\bigskip \bigskip \bigskip \bigskip \bigskip
	{\Large An Email-Centered Approach to Intelligent Task Management Using Crowdsourcing and Natural Language Processing} \\        
	\bigskip \bigskip \bigskip \bigskip
	{\Large John Doe} \\    
	\bigskip
	\begin{figure}[ht]
	\centering \includegraphics[width=0.2\linewidth]{figures/infologo.jpg}
	\end{figure}
	\bigskip 
\end{center}

\vfill

\newpage
\hoffset=5mm
\thispagestyle{empty}

\begin{center}
	\bigskip \bigskip \bigskip 
	\oTUM{6.0cm} \\
	\vspace*{0.8cm}
	{\huge \bf Technische Universität} \\
	\bigskip
	{\huge \bf München} \\
	\bigskip \bigskip \bigskip
	{\huge \bf Fakultät für Informatik} \\
	\bigskip \bigskip \bigskip
	{\Large \bf Master's Thesis in Informatik} \\
	\bigskip \bigskip \bigskip \bigskip \bigskip
	{\Large An Email-Centered Approach to Intelligent Task Management Using Crowdsourcing and Natural Language Processing} \\
	\bigskip \bigskip \bigskip
	{\Large Ein Email-basierter Ansatz für intelligente Aufgabenverwaltung mit Hilfe von Crowdsourcing und Natural Language Processing} \\
	\bigskip
\end{center}
\vfill

\begin{tabular}{ll}
{\Large \bf Author:} & {\Large John Doe} \\\\
{\Large \bf Supervisor:} & {\Large Prof. Dr. Johann Schlichter} \\\\
{\Large \bf Advisor:} & {\Large Dr. Wolfgang Wörndl} \\\\
{\Large \bf Submission:} & {\Large DD.MM.YYYY}
\end{tabular}

\newpage	
\thispagestyle{empty}
\hoffset=0mm
\vspace*{\fill}
\noindent I assure the single handed composition of this master's thesis only supported by declared resources.\\\\
München, DD.MM.YYYY\\\\\\\\\\\\
\noindent \textit{(John Doe)}

\newpage
\thispagestyle{empty}
\null

\newpage
\thispagestyle{empty}
\hoffset=0mm
\section*{Abstract}	
\begin{spacing}{1.2}
\input{abstracte}
\end{spacing}
	
\section*{Inhaltsangabe}
\begin{spacing}{1.2}
\input{abstractd}
\end{spacing}

\newpage
\setcounter{page}{1}
\hoffset=0mm
\bibliographystyle{wmaainf} % quotation style
\setcounter{tocdepth}{3}
\setcounter{secnumdepth}{3}
\fboxsep 0mm

\tableofcontents

\newpage
\setlength{\baselineskip}{3ex}

\begin{spacing}{1.15}
	%\input{chapter1}
	%\input{chapter2}
	%\input{chapter3}
	%\input{chapter4}
\end{spacing}
\newpage
\thispagestyle{empty}
\null

\newpage
\addcontentsline{toc}{chapter}{List of figures}
\listoffigures

%\input{appendices}

\newpage
\thispagestyle{empty}
\null
\newpage
\chapter{Channelmodel}
\begin{figure}
	\centering
	\includegraphics[width=0.7\textwidth]{Channelmodel.PNG}
	\caption{Channelmodel for general Transmitter/Receiver Chain}
	\label{fig:Channelmodel}
\end{figure}

For my purpose we are looking at a simple model which will include a Mapper/Demapper, Encoder/Decode, Interleaving, and the needed channel. We will briefly go over all blocks depicted in the graphic below.

\section{Encoder/Decoder}

For our Encoder/Decoder block we will be looking at a WiMax LDPC code according to the standard IEEE 802.16e. This standard code is used in small and medium distances in urban areas and fits our model quite well. LDPC, which stands for Low Density Parity Check, is a linear error correction code. While it has heavy computing demands, with the growth of computing power it sees more and more use in everyday use. In our case with WiMax we have different given blocksizes ranging from 576 codewords upto 2304. The rates given are 1/2, 2/3, 3/4, and 5/6. We also only look at coding class A for our simulations. 
With a given codeword \textit{x} of length \textit{n} and a generator matrix $G = [I^T|P]$. The parity check matrix texit{H} can now be derived as $H = [-P^T|I\textsubscript{n-k}]$. With the parity check matrix \textit{H} and a code \textit{C} $= xG$ the condition for $cH^t = 0$ must be fulfilled for the codeword to be valid.  
This whole process in MATLAB can be computed with the help of the Coded Modulation Library (CML). For this we have the given function "\textit{InitializeWiMaxLDPC}" to create the parity-check.matrix, "\textit{LdpcEncode}" and "\textit{MpDecode}" to encode and decode our codeword.


\section{Bitinterleave/Deinterleaver}

With the help of bitinterleaving we can avoid any kind of "bursterrors", i.e. we avoid any longer blockerrors. The interleaving in MATLAB is accomplished by creating a random permutation array \textit{k}. The permutation array can be used to randomly shuffle our codeword and also return them back to default at receiver side.

\newpage

\section{Mapper/Demapper}

The mapper or modulation is used to assign a specific codelength a symbol which is transmitted. The symbols are located in a real/imaginary plane. With the distance from the nullpoint of the axis giving the amplitude of our signal and the angle to the real axis the phase shift. 
There are many forms of modulation schemes, with the most common ones being M-PSK, M-FSK, M-AM and M-QAM. For our simulations we will have a further look in QPSK, 16-QAM and 64-QAM, which are depicted below.

\begin{figure}[H]
	\centering
	\includegraphics[width=0.7\textwidth]{Modulation_schemes.PNG}
	\caption{Modulation in I/Q planes for QPSK, 16-QAM and 64-QAM}
	\label{fig:Modulation}
\end{figure}

QPSK, which stands for Quadrature Phase shift keying is one of the easier models we will be looking at. The symbols all share the same amplitude and only differ in their respective phase angle. With the information entropy $S = log\textsubscript{2}(M)$ we can identify the maximum number of bits we can assign in every symbol, with M being the number of symbols in the modulation scheme. So for QPSK the number of bits per symbol amounts to 2.

With M-QAM, Quadrature Amplitude Modulation, we add the phase shift already implemented into QPSK the additional differntiation with the amplitude of symbols. For QAM we send signals which differ in their phase shift and also the amplitude.  For 16-QAM we get a maximum of 3 bits per symbols and for 64-QAM 4 bits per symbol.
The modulation schemes makes it possible to increase our rate of transmission and is used for any kind of practical transmission of information.

\section{Channel}
The channel can be modified in many different ways. We can different sources of noise or fading, which can relate to realworld interferences. Some interferences experienced in real life transmission are e.g. thermal noise, distance, doppler effect and reflection of signals. To approach those kind of interferences there are many different channel models in simulations, like an AWGN-Channel or Rayleigh/Rician fading. We will havea further look into the AWGN-Channel and the Rayleigh fading.

\subsection{AWGN-Channel}
The easiest channel manipulition is to add random gaussian noise to the channel, also commonly known as AWGN-Channel (Additive White Gaussian Noise). Like ths name says we will add noise which is randomly distributed in a gaussian distribution. The probability density function is defined as follows:
\begin{gather*}
f(x|\mu,\sigma^2) = \frac{1}{\sqrt{2\pi\sigma^2}}*e^{-\frac{(x-\mu)^2}{2\mu}}
\end{gather*}
With x being the aquired point, $\mu$ being the mean or expection of the distribution and $\sigma^2$ the variance of the distribution. For our gaussian noise we will take a mean of 0 and a variance of 1, which will simplify further calculations in the following chapters. We will also always look at complex gaussian noise in our simulations. More or less every communication link will have some kind of gaussian noise interference, so we will add the AWGN-Channel to every simulation we run.

\subsection{Rayleigh-Channel}
Another common channel model used in communication theory is Rayleigh fading. Rayleigh fading is used to simulate multipath reception, which means that for a receiver antenna in a wireless link there are many reflected and scattered signals reaching it. This results into construction or destruction of waves. Rayleigh distribution can be defined like this: $R = \sqrt{X^2 + Y^2}$ with X and Y being two independent gaussian distributed random variables. Further calculations will lead to the following pdf:
\begin{gather*}
f(x;\sigma) = \frac{1}{\sigma^2}e^{-\frac{x^2}{2\sigma^2}} 
\end{gather*} 

	
\newpage
\chapter{Capacity in a AWGN channel}

We will now look into the maximum capacity we can achieve for our communication model in Chapter 2 with added AWGN noise.

\section{Capacity}

In general capacity \textit{C} can be defined as the rate \textit{R} at which information can be reliably transmitted over a channel, which means as long $R \leq \ $C we can achieve a transmission without errors even with noise. All the capacities we wil be looking at will be for complex channel models.

Capacity is the maximum Entropy or mutual Information given by the two random variables 
% Generierung des Literaturverzeichnisses
%\bibliography{/path/to/your/.bib/file}

\end{document}
